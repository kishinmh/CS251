\documentclass[]{article}

\begin{document}

\thispagestyle{empty}
\begin{center}
 {\scshape \Large \textbf{ Movie Review Anand and Golmaal}\\} {\scshape \large \textbf {CS251 Assignment}\\}
 {\scshape \large \textbf{Kumar Kshitij Patel (150348)\\
 Piyush Bansal (150488)\\
 Adwait Vyawahare (13807)}
 }
 \end{center}
 
\smallskip

\hrule

\bigskip

\Large {\textbf{ANAND\\}}
\large{
Anand('Happiness') is as happy a movie about terminal illness as you’re likely to find. Nevertheless, this classic Hindi film raises interesting questions about how healthy people should interact with the sick, and whether preemptive grief makes things worse.

The story unfolds through the diary entries of Dr. Bhaskar Banerjee (Amitabh Bachchan), a thirty-year-old cancer specialist who’s already overwhelmed by the amount of death he’s seen in his career. His own sense of futility — “Oh, God! Such frightening helplessness.” — makes him persnickety with his patients. He sees no point in giving them false hope.

Being a bachelor, Bhaskar is the logical choice to host Anand (Rajesh Khanna), a friend of a friend who wants to tour Bombay. Bhaskar and his friend and fellow doctor, Prakash (Ramesh Deo), are tasked with treating Anand, who’s suffering from terminal intestinal cancer.

Anand talks non-stop, joking with friends and strangers alike. Such relentless cheerfulness would normally grate on a man as taciturn as Bhaskar, but the doctor sees the melancholy behind Anand’s permanent grin.

The most obvious message in the movie is to be “live each day to the fullest,” . Anand knows he’s dying, yet he chooses to remain upbeat. He’d rather have fun while he’s here, and he wants his friends to be happy when they’re with him.

At one point, Anand threatens to leave rather than endure anymore of Bhaskar’s forlorn looks. The film makes a persuasive case that the ill have the right to dictate how they are treated by others. Just because our own inclination may be to mourn what we know we will lose, it doesn’t mean we should subject the person who is dying to those feelings.

More than forty years after its original release, the shots of Mumbai (then Bombay) make the city look like a very cool place to be in the early 1970s. Bachchan looks dashing, and the furnishings in Bhaskar’s house are fabulous.

The only aspects of Anand that don’t really translate to the modern-day are its aggressive musical cues (though plenty of directors still rely on them). Any time Anand gasps for breath, the sounds of an orchestra blare to ensure the audience knows that the moment is significant. The effect is more shocking than instructive, especially in cases where the music intrudes on an otherwise quiet scene.

The relatively small cast delivers great performances, particularly in regard to the way they react to Anand’s sickness. Bhaskar’s beloved, Renu (Sumita Sanyal), is rock steady, while Prakash’s wife, Suman (Seema Deo), loses her courage. Bachchan keeps the core of Bhaskar intact, opening him up to Anand — and the world — gradually.

While playing a character who’s described as a “tornado,” Rajesh Khanna carefully ensures that Anand feels realistic, rather than like some outrageous film creation. Anand is fun-loving, but not a clown. He has moments of melancholy, but he’s not harboring a dark secret. He really is just a guy who wants to be happy while he can.

Thanks to fine performances and a charming lead character, Anand is good for watching in cinemas.\\}

\Large {\textbf{GOLMAAL\\}}
\large{
Amol Palekar was the common man’s hero, living his dreams on the screen. So middle class, in appearance and his roles, yet strikingly unique and rich in his acting repertoire! He would not miss the 9.10 local to woo his lady love in “Baaton Baaton Mein” or sing his way into the village belle’s heart in “Chitchor” or take lessons in courtship to win his battle in “Chhoti Si Baat”.

But here he had the daunting task of performing a double role, one man with two faces — one, a simple soul and the other, pretentious; one, an introvert and the other, garrulous. It was tough but then Palekar, a product of Marathi theatre, slipped into the director’s demands with an ease that belied his personality.

Palekar, having been a painter and staged exhibitions of his work, didn’t find it difficult to surmise his success when enacting roles that many dreaded. After he was not chasing girls around the trees or bashing up the villain and his army singlehandedly. Palekar epitomised you and me, close to everyday reality of life.

In this film, Palekar is Ram Prasad Sharma and his ‘twin’ Laxman Prasad Sharma. He is a sports and music lover, day dreamer too. He sets the alarm at 5.30 a.m. to listen to Test commentary, expecting Sunil Gavaskar, Viswanath, Bishan Bedi, Chandrashekhar and Mohinder Amarnath to fox the Australians; he is also excited with Black Pearl ‘Pele’ coming with Cosmos to play Mohun Bagan and also follow Islauddin and Samiullah torment the Indians on the hockey field. One has hardly known a film hero with such diverse interest in sport.

When Hrishikesh Mukherjee picked Palekar for the role in “Gol Maal”, he was investing in a brilliant actor, who matched the other strong personality in the film, veteran Utpal Dutt. The two struck a brilliant rapport, bringing composed but compelling images of their characters. Dutt played an industrial, stuck with traditional and strong pre-conceived ideas of a successful person. Palekar was the refreshing contrast to the senior actor. Both, least surprisingly, won the Filmfare Award for their show; Palekar for best actor and Dutt for comic role.

It was a tribute to Palekar’s work that he beat Amtitabh Bachchan and Rajesh Khanna to the Filmfare Award. It remains a highpoint of his career and acknowledged by many critics in later years. This was a double role with enormous pressure. Ashok Kumar, Dev Anand and Dilip Kumar had handled the responsibility with some stunning success but the difference was notable. The three stalwarts had two different characters to portray but Palekar was one man playing Ram Prasad Sharma and Laxman Prasad Sharma. The audience knew all along that Ram Prasad and Laxman Prasad were the same. A mere moustache differentiated the two.

It may appear strange that a man could transform into another personality by just sporting a thin moustache. If a celebrated Hrishikesh Mukherjee could be convinced the rest did not matter. Ram Prasad sports a moustache only to win the heart of Dutt, his demanding employer. Dutt firmly believes that clean shaven men are not to be trusted. Ram Prasad has little choice but impersonate; to Palekar’s credit, he delivers in style, hard to choose which is the better depiction, Ram Prasad with moustache or his clean-shaven ‘brother’ Laxman Prasad.

It is a classical comedy, not loud at any point, the humour subtle and intelligent. Ram Prasad wants a job and finds it courtesy doctor uncle (David), who directs him to Bhavani Shankar (Utpal Dutt). Bhavani has a daughter Urmila (Bindiya Goswami) and can’t stand youth in modern attire. Bhavani loves sport but not youngsters with similar vocation. Urmila thinks differently.

The kurta-pyjama clad Ram Prasad, guided by his actor-friend Deven (Deven Verma), makes an impression on Bhavani, who sees a prospective son-in-law in him. But the original Ram Prasad, in colourful clothing and sporting sun glares, is spotted by Bhavani at an India-Pakistan hockey Test match. Ram Prasad convinces Bhavani it was his twin brother who caused the confusion. Bhavani falls into the trap which also includes Dina Pathak donning the role of Ram Prasad’s mother.

Bhavani offers Laxman Prasad the job of teaching music to Urmila, who falls in love with her teacher. When Bhavani discovers the truth, a hilarious chase ends up in the police station. Om Prakash and Utpal Dutt liven up the closing chapter of the film. Om Prakash shouting “Pooolice Officer” and Dutt retorting with a “Foolish Officer” refrain is one of the enjoyable moments of the film.

The film highlights the directorial excellence of Hrishikesh Mukherjee. He was a complete filmmaker, who gave tear-jerkers like “Anand” and “Mili” and also sterling comedies like “Chupke Chupke”, “Guddi”, “Khubsoorat” and “Bawarchi”. The background score by Rahul Dev Burman, and of course the four songs, stand out in ‘Gol Maal”, the best being “Aane Wala Pal Jaane Wala Hai”, a glowing tribute to Gulzar, who won the best lyricist award for this lingering Kishore Kumar number. It is an Amol Palekar movie all the way with Utpal Dutt brilliant as ever. It remains a film worth watching many a time!}

\end{document}
